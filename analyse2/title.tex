\begin{titlepage}
%~ % \newcommand{\HRule}[2]{\centering\rule{#1}{#2}}
%~ \newlength{\logowidth}
%~ \setlength{\logowidth}{2.5cm}
%~ \newlength{\logohspace}
%~ \setlength{\logohspace}{\linewidth}
%~ \addtolength{\logohspace}{-3\logowidth}
%~ 
%~ \begin{center}
%~ % Upper part of the page
%~ % \begin{minipage}{0.4\textwidth}
%~ %   \begin{flushleft} 
%~ %     \includegraphics[width=4cm]{./figure/Logo_ETIS.png}\\[1cm]
%~ %   \end{flushleft}
%~ % \end{minipage}
%~ % \begin{minipage}{0.4\textwidth}
%~ %   \begin{flushright}
%~ %    \includegraphics[width=3cm]{./figure/Logo_CNRS.png}\\[1cm]
%~ %   \end{flushright}
%~ % \end{minipage}
%~ % \begin{minipage}{0.4\textwidth}
%~ %  
%~ % \end{minipage}
    	%~ %%%%%%%%%%%% Logos
	%~ %\includegraphics[height=2.5cm]{./stage_fig/Logo_ETIS.eps}\hspace{\logohspace}
	%~ \includegraphics[height=2.5cm]{./stage_fig/Logo_ETIS}\hspace{\logohspace}
	%~ %\includegraphics[height=2.5cm]{./stage_fig/Logo_ENSEA.eps} 
	%~ \includegraphics[height=2.5cm]{./stage_fig/Logo_ENSEA} 
%~ 
%~ % \vspace{5cm}
%~ \vfill
%~ \textsc{\LARGE Rapport de stage}\\[1.5cm]
%~ % Title
%~ \HRule \\[0.4cm]
%~ { \huge \bfseries Codes LDPC non-binaires pour compression avec information adjacente}\\[0.4cm]
%~ \HRule \\[1.5cm]
%~ 
%~ % Author and supervisor
%~ \begin{minipage}{0.4\textwidth}
%~ \begin{flushleft} \large
%~ Anne \textsc{SAVARD}
%~ \end{flushleft}
%~ \end{minipage}
%~ \begin{minipage}{0.4\textwidth}
%~ \begin{flushright} \large
%~ \emph{Encadrant:} \\
%~ Claudio \textsc{WEIDMANN}
%~ \end{flushright}
%~ \end{minipage}
%~ 
%~ \vfill
%~ 
%~ % Bottom of the page
%~ {\large \today}
%~ 
	%~ %%%%%%%%%%%% Logos
	%~ \includegraphics[height=2.5cm]{./stage_fig/Logo_UCP}\hspace{\logohspace}
	%~ %\includegraphics[height=2.5cm]{./stage_fig/Logo_UCP.eps}\hspace{\logohspace}
	%~ %\includegraphics[height=2.5cm]{./stage_fig/Logo_CNRS.eps} 
	%~ \includegraphics[height=2.5cm]{./stage_fig/Logo_CNRS} 
	%~ %%%%%%%%%%%%
%~ \end{center}
\thispagestyle{empty}
\newcommand{\HRule}[2]{\centering\rule{#1}{#2}}
% \newlength{\logowidth}
% \setlength{\logowidth}{2.5cm}
% \newlength{\logohspace}
% \setlength{\logohspace}{\linewidth}
% \addtolength{\logohspace}{-3\logowidth}

Lycée Blaise Pascal
\vspace{20em}
\begin{center}
%Règles de dépôt : vous devrez déposer un document unique comprenant :
%une page de garde avec votre nom et votre prénom, le niveau de la classe,
%l’indication et l’intitulé de l’unité d’enseignement et l’année universitaire en
%cours. Une introduction d’une quinzaine de lignes pour indiquer le contexte
%(situation géographique, effectif...) La rédaction de votre analyse réflexive en
%réponse aux questions de votre tuteur ESPE (3/4 pages maximum). Des
%annexes si nécessaire.

	\HRule{\linewidth}{0.5mm}\\
	\vspace{-0.35cm}
	\HRule{\linewidth}{0.3mm}\\
	\textbf{
	\LARGE
	Analyse réflexive (2)\\
	}
	\large
	sur un séance d'Enseignements Technologiques Transversaux\\
	en première STI2D
	\HRule{\linewidth}{0.3mm}\\
	\vspace{-0.45cm}
	\HRule{\linewidth}{0.5mm}

	\vspace{10em}

	par\\
	Laurent Fiack\\

	\vspace{3em}
	pour la validation du parcours adapté en SII\\
	\vspace{3em}
	\today{}

\end{center}

\end{titlepage}    
