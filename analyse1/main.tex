\documentclass[pdftex,a4paper,12pt]{article}

\usepackage[francais]{babel}
\usepackage{epsfig}
\usepackage{amsfonts}
\usepackage[utf8x]{inputenc}
%\usepackage{palatino}
\usepackage{mathptmx}
\usepackage{geometry}
\usepackage{float}
\usepackage{amsmath}
\usepackage[babel=true]{csquotes}
\usepackage{graphicx}
\usepackage{algorithm}
\usepackage{algorithmic}
\usepackage{listings}
\usepackage{color}
\usepackage[squaren, Gray, cdot]{SIunits}
\usepackage[square]{natbib}
\usepackage{eurosym}
\usepackage{listings}

\geometry{a4paper,headheight=16pt,tmargin=20mm,
          bmargin=20mm,lmargin=20mm,rmargin=20mm}
%\geometry{height=23cm}
%\geometry{hmargin={3cm,2cm}}

\usepackage{caption}
\usepackage{subcaption}

\usepackage{minitoc}
%\setcounter{minitocdepth}{3}
\mtcselectlanguage{french} 

\usepackage{hyperref}

\usepackage{tabularx}
\usepackage{array}
\usepackage[normalem]{ulem}
\newcolumntype{L}[1]{>{\raggedright\let\newline\\\arraybackslash\hspace{0pt}}m{#1}}
\newcolumntype{C}[1]{>{\centering\let\newline\\\arraybackslash\hspace{0pt}}m{#1}}
\newcolumntype{R}[1]{>{\raggedleft\let\newline\\\arraybackslash\hspace{0pt}}m{#1}}

\usepackage[printonlyused,withpage]{acronym}

\newcommand{\figpath}{figures}

%Règles de dépôt : vous devrez déposer un document unique comprenant :
%une page de garde avec votre nom et votre prénom, le niveau de la classe,
%l’indication et l’intitulé de l’unité d’enseignement et l’année universitaire en
%cours. 
%Une introduction d’une quinzaine de lignes pour indiquer le contexte
%(situation géographique, effectif...) 
% La rédaction de votre analyse réflexive en
%réponse aux questions de votre tuteur ESPE (3/4 pages maximum). Des
%annexes si nécessaire.

%\sloppy
\begin{document}
	\begin{titlepage}
%~ % \newcommand{\HRule}[2]{\centering\rule{#1}{#2}}
%~ \newlength{\logowidth}
%~ \setlength{\logowidth}{2.5cm}
%~ \newlength{\logohspace}
%~ \setlength{\logohspace}{\linewidth}
%~ \addtolength{\logohspace}{-3\logowidth}
%~ 
%~ \begin{center}
%~ % Upper part of the page
%~ % \begin{minipage}{0.4\textwidth}
%~ %   \begin{flushleft} 
%~ %     \includegraphics[width=4cm]{./figure/Logo_ETIS.png}\\[1cm]
%~ %   \end{flushleft}
%~ % \end{minipage}
%~ % \begin{minipage}{0.4\textwidth}
%~ %   \begin{flushright}
%~ %    \includegraphics[width=3cm]{./figure/Logo_CNRS.png}\\[1cm]
%~ %   \end{flushright}
%~ % \end{minipage}
%~ % \begin{minipage}{0.4\textwidth}
%~ %  
%~ % \end{minipage}
    	%~ %%%%%%%%%%%% Logos
	%~ %\includegraphics[height=2.5cm]{./stage_fig/Logo_ETIS.eps}\hspace{\logohspace}
	%~ \includegraphics[height=2.5cm]{./stage_fig/Logo_ETIS}\hspace{\logohspace}
	%~ %\includegraphics[height=2.5cm]{./stage_fig/Logo_ENSEA.eps} 
	%~ \includegraphics[height=2.5cm]{./stage_fig/Logo_ENSEA} 
%~ 
%~ % \vspace{5cm}
%~ \vfill
%~ \textsc{\LARGE Rapport de stage}\\[1.5cm]
%~ % Title
%~ \HRule \\[0.4cm]
%~ { \huge \bfseries Codes LDPC non-binaires pour compression avec information adjacente}\\[0.4cm]
%~ \HRule \\[1.5cm]
%~ 
%~ % Author and supervisor
%~ \begin{minipage}{0.4\textwidth}
%~ \begin{flushleft} \large
%~ Anne \textsc{SAVARD}
%~ \end{flushleft}
%~ \end{minipage}
%~ \begin{minipage}{0.4\textwidth}
%~ \begin{flushright} \large
%~ \emph{Encadrant:} \\
%~ Claudio \textsc{WEIDMANN}
%~ \end{flushright}
%~ \end{minipage}
%~ 
%~ \vfill
%~ 
%~ % Bottom of the page
%~ {\large \today}
%~ 
	%~ %%%%%%%%%%%% Logos
	%~ \includegraphics[height=2.5cm]{./stage_fig/Logo_UCP}\hspace{\logohspace}
	%~ %\includegraphics[height=2.5cm]{./stage_fig/Logo_UCP.eps}\hspace{\logohspace}
	%~ %\includegraphics[height=2.5cm]{./stage_fig/Logo_CNRS.eps} 
	%~ \includegraphics[height=2.5cm]{./stage_fig/Logo_CNRS} 
	%~ %%%%%%%%%%%%
%~ \end{center}
\thispagestyle{empty}
\newcommand{\HRule}[2]{\centering\rule{#1}{#2}}
% \newlength{\logowidth}
% \setlength{\logowidth}{2.5cm}
% \newlength{\logohspace}
% \setlength{\logohspace}{\linewidth}
% \addtolength{\logohspace}{-3\logowidth}

Lycée Blaise Pascal
\vspace{20em}
\begin{center}
%Règles de dépôt : vous devrez déposer un document unique comprenant :
%une page de garde avec votre nom et votre prénom, le niveau de la classe,
%l’indication et l’intitulé de l’unité d’enseignement et l’année universitaire en
%cours. Une introduction d’une quinzaine de lignes pour indiquer le contexte
%(situation géographique, effectif...) La rédaction de votre analyse réflexive en
%réponse aux questions de votre tuteur ESPE (3/4 pages maximum). Des
%annexes si nécessaire.

	\HRule{\linewidth}{0.5mm}\\
	\vspace{-0.35cm}
	\HRule{\linewidth}{0.3mm}\\
	\textbf{
	\LARGE
	Analyse réflexive (2)\\
	}
	\large
	sur un séance d'Enseignements Technologiques Transversaux\\
	en première STI2D
	\HRule{\linewidth}{0.3mm}\\
	\vspace{-0.45cm}
	\HRule{\linewidth}{0.5mm}

	\vspace{10em}

	par\\
	Laurent Fiack\\

	\vspace{3em}
	pour la validation du parcours adapté en SII\\
	\vspace{3em}
	\today{}

\end{center}

\end{titlepage}    

%Une introduction d’une quinzaine de lignes pour indiquer le contexte
%(situation géographique, effectif...) 
	\section{Introduction}
	\subsection{Contexte général}
	Jeune professeur agrégé stagiaire, j'ai pris mes fonctions il y a environ deux mois au Lycée Blaise Pascal de Rouen, 
	situé au c\oe{}ur de la ville sur la rive gauche de la Seine.
	Ce lycée est assez imposant puisqu'il accueille environ 1200 élèves, près de 120 enseignants et plus de 60 personnels non-enseignants.\\
	
	Au cours de cette année de titularisation, deux classes me sont confiées.
	La première est l'une des 9 secondes où j'interviens une heure et demie en enseignement d'exploration \og{}Sciences de l'Ingénieur\fg{}.

	La seconde est l'une des 4 premières STI2D où je co-anime les séances d'applications pratiques d'Enseignements Technologiques Transversaux (abbrégés ETT) 
	avec ma collègue Nathalie Lebarbier pour un volume de 5 heures hebdomadaires.
	J'interviens également une heure par semaine en classe entière, dans la même matière, ainsi qu'une heure en accompagnement personalisé.\\

	À travers ce document, je vais tenter de développer une analyse réflexive sur la démarche pédagogique que j'ai mise en place depuis ma prise de fonction,
	en m'appuyant sur les remarques formulées par mes tuteurs terrain et ESPE au cours de la visite conseil 
	d'une des séances d'application pratique le lundi 2 octobre 2017.

	\subsection{La séance en question}
	La séance visitée était la dernière d'une séquence de trois activités d'étude de dossier pour découvrir du langage SysML.
	Au cours de cette séquence, les élèves ont d'abord découvert quelques notions par l'étude d'un scooter sous-marin lors de la première activité. 
	Ils ont ensuite approfondi ces notions dans l'étude d'une voiture télécommandée avec la deuxième activité
	pour finalement aborder tous les diagrammes du langage SysML au programme de la STI2D pendant la troisième activité.

	Cette dernière étude de dossier concerne un aménagement urbain : une borne escamotable.
	Au cours de cette activité, les élèves disposaient d'une page web agrégeant les ressources ainsi que
	d'un document au format Word pour répondre à une série de questions. 
	
	Le choix de réaliser des activités spécifiquement centrées sur le langage SysML est discutable. 
	Si ce choix a été fait, c'est par demande du collègue professeur de spécialité SIN car il se base sur ce prérequis pour lancer ses activités.
	À mon sens, il serait plus judicieux de distiller la découverte du SysML par l'étude de systèmes tout au long de l'année 
	dans le cadre d'activités sur d'autre thématiques.

	\subsection{Les points à analyser}
	Parmis les remarques formulées, quatre ont été mise en avant par mes tuteurs:
	\begin{enumerate}
		\item Au cours de la séance visitée, les élèves étaient chacun face à un ordinateur et travaillent avec un questionnaire, des ressource et le document réponse. 
			Pourquoi ne pas les faire travailler en binôme ?
		\item Organisation de la fiche professeur de préparation de séance, listant de manière plus où moins détaillée l'ensemble des points à aborder.
		\item Ajouter plus de concret, plus d'interactivité classe/professeur, et varier les supports.
		\item Approfondir et tester une idée d'activité \og{}fil rouge\fg{} sur l'année ou le trimestre pour les élèves finissant le travail en avance.
	\end{enumerate}

	Dans cette analyse, je me concentrerais sur les premier et troisème points, qui s'articulent autour de la gestion de la classe 
	et dont les éléments de réponse me semblent complémentaires.

	Le point numéro 2 quant à lui concerne l'organisation des séances et évolue au fil des séances.
	Enfin, le quatrième point mérite à mon sens une étude sur un plus long terme, je souhaite donc le développer au cours de mon \og{}Projet Collaboratif Innovant\fg{}.

	\section{Travailler en monôme ou en binôme ?}
	\subsection{Une question d'organisation}
%* Dans la salle, il y a un ordinateur par élève, les élèves se mettent naturellement chacun devant une machine. Il se sont placés par affinité.
	La salle d'activité pratique est organisée en ilôts de quatre postes informatiques, et d'autant de chaises.
	Il y a suffisamment de postes pour que chaque élève puisse avoir accès au sien.
	Les élèves se sont naturellement placés devant une machine, l'organisation dans les ilôts s'est faite par affinité.
%* Il y a 4 machines par ilôt, donc max 4 élèves. Les élèves au sein d'un ilôt ont chacun leur machine, mais peuvent facilement communiquer (et y sont autorisés).
	Bien que chaque élève soit seul en face de sa machine, ils peuvent aisément communiquer entre eux au sein d'un ilôt, et y sont autorisés.\\

%* Les premières séances ont consisté en des études de dossier ayant pour objectif de découvrir le langage SysML.
%Ces travaux ne nécessitent pas de matériel, hormis les systèmes étudiés dans les différentes activités.
%Ce matériel n'est pas présent en nombre suffisant pour que chaque ilôt ou binôme dispose du sien.
	Les premières séances ont consisté en des études de dossier ayant pour objectif de découvrir le langage SysML.
	Ces travaux ne nécessitent pas de matériel, hormis les systèmes étudiés dans les différentes activités.
	Ce matériel n'est pas disponible en nombre suffisant pour que chaque ilôt ou binôme dispose du sien et circule donc dans la classe pendant l'activité.

%	* Les activités suivantes, ayant eu lieu entre la visite et la rédaction de ce document, abordent les notion de codage de l'information et du numération.
%Ces activités ne nécessitent pas plus de matériel et les élèves travaillent toujours face à leur machine.
%	* Lors des séances futures, en énergie notamment, les élèves seront amenés à plutôt manipuler les maquettes en groupe.
	Les activités suivantes, ayant eu lieu entre la visite et la rédaction de ce document, abordent les notion de codage de l'information et du numération.
	Ces activités ne nécessitent pas plus de matériel et les élèves travaillent toujours face à leur machine.
	Lors des séances futures, en énergie notamment, les élèves seront amenés à plutôt manipuler les maquettes en groupe.\\

%* Lors d'une séance, nous sommes deux professeurs pour une classe entière. 
%Je m'adapte aux méthodes de ma collègue, qui agit selont le fonctionnement de l'établissement.
	Au cours de la séance, je m'adapte aux méthodes de ma collègue Nathalise Lebarbier, avec qui je co-anime la séance.
	En tant que nouvel arrivant, je m'adapte à sa manière de travailler, fidèle au fonctionnement de l'établissement.

	\subsection{Le travail en monôme et l'autonomie}

%Elle préfère voir les élèves travailler en monome, pour plusieurs raisons :
%	* La quantité de bruit est moins élevée. L'augmentation naturelle du volume sonore des élèves qui travaillent en groupe incite les élèves à parler encore plus fort. 
%Lorsqu'ils travaillent en binôme, il faut régulièrement les rapeller à l'ordre.
%	* L'évaluation, par compétences ou non, est plus aisée.
%	* Lors du baccalauréat, les élèves seront seuls face à leur copie. 
%Même si cette affirmation n'est pas fausse en soit, je ne suis pas tout à fait d'accord avec cet argument: 
%même s'il reste un objectif indispensable, le bac n'est, à mon sens, pas le seul objectif de la formation: 
%La formation vise également à engager les élèves dans une poursuite d'études, pour qu'ils puissent, par la suite, s'intégrer dans le monde de l'entreprise.
%Les jeunes bacheliers seront alors bien évidemment amenés à collaborer avec leurs pairs, d'abord dans l'enseignement supérieur, 
%et à plus forte raison dans leur carrière professionnelle.
	Au cours de discution avec mes collègues, plusieurs motifs sont énoncés pour justifier le travail en autonomie des élèves.
	D'abord, de manière assez pragmatique, la quantité de bruit dans la salle est moins élevée. 
	L'augmentation naturelle du volume sonore des élèves qui travaillent en groupe incite leurs camarades à parler encore plus fort, phénomène qui s'amplifie rapidement. 
	Lorsqu'ils travaillent en binômes, il faut régulièrement rapeller les élèves à l'ordre.

	Ensuite, quand les élèves travaillent seuls, il est plus aisé d'évaluer leur niveau et de remédier à des difficultés. 
	Au contraire, lorsqu'ils travaillent en groupe, certaines difficultés peuvent passer inaperçu.

	Enfin, lorsqu'ils passeront les épreuves du baccalauréat, les élèves seront seuls face à leur copie.
	Même si cette affirmation n'est pas fausse en soit, j'aimerais ajouter un peu de nuances à cet argument.
	Même s'il reste un objectif indispensable, le bac n'est, à mon sens, pas le seul objectif de la formation.
	Celle-ci vise en effet également à engager les élèves dans une poursuite d'études, pour qu'ils puissent, par la suite, s'insérer dans le monde de l'entreprise.
	Les jeunes bacheliers seront alors bien évidemment amenés à collaborer avec leurs pairs, 
	d'abord dans l'enseignement supérieur, et à plus forte raison dans leur carrière professionnelle.\\

%* En faisant travailler les élèves en monôme, j'ai pu constater que certains élèves ne sont pas familiarisé à l'utilisation de systèmes informatiques. 
%Je préfère m'assurer qu'ils aient tous l'opportunité de le manipuler, car je suis convaincu que ce n'est qu'en manipulant qu'ils pourront mieux s'y familiariser.
	Finalement, en faisant travailler les élèves en monôme, j'ai pu constater que certains élèves ne sont pas familiarisé à l'utilisation de systèmes informatiques.
	Je préfère m'assurer qu'ils aient tous l'opportunité de le manipuler, car je suis convaincu que ce n'est qu'en manipulant qu'ils pourront mieux combler ces lacunes.

	\subsection{Le travail en groupe et la collaboration}
	Il a été discuté plus haut de la nécessité de préparer les élèves à la poursuite d'études 
	pour se diriger ensuite vers une future intégration dans le monde professionnel.
%travail en projet, répartition des tâches, organisation
%Cependant, la notion de projet est plus présente en spécialité.
	Ainsi nous devons préparer les élèves à travailler de manière collaborative sur des projets communs et leur apprendre à s'organiser et à répartir leurs tâches.
	Selon les référentiels, cependant, les travaux en projets sont plutôt présents en spécialité.\\

%* La mise en binôme de deux élèves est souvent bénéfique. Les deux élèves n'ont pas les mêmes niveaux de compréhension dans tous les domaines.
%L'élève ayant un meilleur niveau de compréhension des notions abordées par l'activité en cours va souvent ré-expliquer ces notions à son binôme, 
%les rôles pouvant s'inverser en fonction des activités.
	Les élèves ont souvent de niveaux de compréhension assez différents en fonction des domaines de compétences.
	Ainsi, la mise en binôme des élèves est souvent bénéfique : 
	un élève ayant un meilleur niveau de compréhension des notions abordées par l'activité en cours va souvent ré-expliquer ces notions à son camarade,
	les rôles pouvant s'inverser en fonction des activités.

%Cette reformulation est bénéfique pour les deux membres, d'une part un élève ayant des difficultés pourra tirer bénéfice de la reformulation d'un de ces camarades.
%D'autre part, l'élève qui reformule les notions s'assure de mieux les comprendre et de mieux les retenir.
%* Il faut toutefois veiller, notamment lors d'une grande différence de niveaux, que l'élève ayant le plus de facilité ne réalise pas le travail à la place de son camarade.
	Cette reformulation est bénéfique pour les deux élèves.
	D'une part l'élève le moins à l'aise avec le sujet pourra tirer bénéfice de la reformulation d'un de ces camarades.
	D'autre part, l'élève qui reformule les notions s'assure de mieux les comprendre et surtout de mieux retenir ces dernières.

	Il faut toutefois veiller, face une grande différence de niveaux notamment,
	que l'élève ayant le plus de facilité ne réalise pas le travail à la place de son camarade, 
	ce qui empècherais tout appropriation des notions par l'élève en difficulté.

	\subsection{Quelle organisation?}
	En conclusion, nous avons vu que le choix de placer chaque élève face à un ordinateur provient d'abord d'une question d'organisation, 
	du fonctionnement du lycée, et de la présence de matériel suffisant.\\

%Même s'il a été admis que, dans le monde de l'entreprise, on est amené à collaborer avec ses pairs, il est rare, dans l'industrie de travailler en binôme sur un même poste. Les élèves doivent ainsi, selon moi, apprendre à collaborer, tout en fournissant un travail personnel.
	Il a été énoncé plus haut que les élèves seront amenés à collaborer avec leurs pairs à l'issue de leur cursus de formation, dans le monde de l'entreprise.
	Il me semble cependant qu'il soit rare dans l'industrie de travailler en binôme sur un même poste.
	Au contraire, dans les société d'informatique notamment, les employés travaillent le plus souvent en autonomie sur leur poste, 
	font un point régulier, et s'entraident au besoin.
	
	Les élèves doivent ainsi, selon moi, apprendre à collaborer, tout en fournissant un travail personnel.\\
	
	Si j'admet que le travail en groupe peut être bénéfique, il ne suffit pas de mettre les élèves en binômes pour qu'émerge un travail de qualité.
	Dans la section suivante nous aborderons entre autres quel rôle l'enseignant peut avoir dans cet apprentissage du travail en groupe.

	\section{Plus de concret, d'interactions et des supports plus variés}
%* Regardant le concret:
	\subsection{Travailler avec un objet tangible}
%	* Les deux première activités/étude de dossier (avant celle visitée, donc) avaient un objet réel, scooter sous-marin (1 pour tout le groupe), et voiture télécommandée (2 pour le groupe).
%	* La troisième étude de dossier (celle visitée) concernait un amménagement urbain: une borne escamotable. À la place d'une maquette à manipuler, les élèves avaient deux vidéos à disposition.
	L'activité visitée consistait en une étude de dossier sur une borne escamotable en milieu urbain.
	Les élèves ne disposait que de vidéos pour se représenter le système. 
	Il aurait pu être plus intéressant de leur permettre de manipuler une maquette représentant le système.

	Lors des deux précédentes activités, les élèves avaient pu observer voire manipuler les objets étudiés.\\

%	* Les activités suivantes sur le codage de l'information et la numération s'effectuent sans objet réel autre que l'ordinateur. 
%Cependant, les élèves ont été amenés à travailler sur un fichier image, à l'aide d'un éditeur hexadécimal et d'un visionneur d'images, apportant un peu plus de concrêt.
	Les activités suivantes -- ayant eu lieu entre la visite conseil et la rédaction de ce document -- ont traité de la numération et du codage de l'information.
	Elle ont également été réalisées sans autre objet réel que l'ordinateur.
	
	Les élèves ont cependant été amenés à travailler sur un fichier image, à l'aide d'un éditeur hexadécimal et d'un visionneur d'images, 
	apportant un peu plus de concret.\\

%	* Les thématiques abordées jusqu'ici concernait principalement des aspects d'information, pour lesquels il est plus difficile de trouver des objets tangibles concrêts à manipuler. Lors des séances à venir sur l'énergie ou la matière, les élèves seront plus amenés à manipuler.
	Les thématiques abordées jusqu'ici concernait principalement des aspects d'information, 
	pour lesquels il est plus difficile de trouver des objets tangibles concrêts à manipuler. 
	
	Lors des séances à venir sur l'énergie ou la matière, les élèves seront plus amenés à manipuler des systèmes de mesure, ou des objets plus concrets.\\

%* Comment alors amener plus de concret sur l'axe information? 
%	* Codage/numération : utilisation d'un afficheur LCD alphanumérique (code ASCII), afficheurs 7 segments (code BCD).
%	* Logique combinatoire : Manipulation du tableau de commande d'un banc moteur (lien avec énergie), 
%étude d'un "radio-réveil" déployé sur circuit logique reconfigurable (eg: [DE10-Lite](http://www.terasic.com.tw/cgi-bin/page/archive.pl?Language=English&CategoryNo=234&No=1021))(lien avec codage).
%	* Modèles algorithmiques : Lego mindstorm, arduino/blockly
	On peut alors se poser la question suivante : comment amener plus de concret sur l'axe information?
	Cette année, l'axe information est abordé a travers trois grandes thématiques, la numération et le codage de l'information, la logique combinatoire 
	et enfin les modèles algorithmiques.

	Le codage de l'information pourrait être illustré par l'utilisation d'afficheurs LCD alphanumériques, lors de manipulation du code ASCII.
	On pourrait de manière analogue utiliser des afficheurs 7 segments pour donner plus de sens au code BCD.
	Enfin, lors de la première activité sur le codage de l'information, les élèves ont étudié une horloge binaire 
	\footnote{\textbf{Horloge binaire:} \url{http://www.semageek.com/bbc-horloge-binaire-gante/}}.
	Il serait assez aisé de mettre en place une maquette de cette horloge dans notre salle.

	La logique combinatoire peut être abordée par la manipulation du tableau de commande d'un banc moteur et faire le lien avec l'énergie.
	On pourrait également étudier un \og{}radio-réveil\fg{}déployé sur circuit logique reconfigurable, comme le DE10-Lite de Terasic 
%	\footnote{\url{http://www.terasic.com.tw/cgi-bin/page/archive.pl?Language=English&CategoryNo=234&No=1021}}
	\footnote{\textbf{DE10-Lite:} \url{http://www.terasic.com.tw/cgi-bin/page/archive.pl?CategoryNo=234&No=1021}}
	et ainsi faire le lien avec le codage de l'information vu plus haut.

	Enfin, le lien avec les modèles algorithmiques est peut être le plus simple. 
	Il existe, dans l'environnement didactique, de nombreux objets programmables accessibles pour les élèves.
	On peut citer par exemple les Lego Mindstorms ou encore de nombreux robots basés sur la carte Arduino, programmable avec le langage graphique blockly.

%* Regardant l'interactivité classe/professeur
	\subsection{Interactivité entre la classe et le professeur}
%	* L'activité visitée était la dernière d'une séquence de trois activités. 
%Le groupe étant très hétérogène, j'ai proposé aux élèves les plus rapides de prendre de l'avance, tandis que j'ai préféré laisser les élèves moins rapides finir les activités précédentes.
%Ainsi, au début de la troisième séance, certains élèves avaient presque terminé la troisième activité alors que d'autres en étaient encore au début de la deuxième.
%J'ai finalement passé la séance à faire du cas par cas et a répéter beaucoup d'explications.
%	* Mes principales difficultés dans cette interactivité proviennent de la gestion de l'hétérogénéité du groupe.
%* Comment faire face à l'hétérogénéité du groupe:
%	* Ne pas laisser les plus rapides prendre trop d'avance : ne pas passer à l'activité suivante. 
%Leur donner trop d'exercices supplémentaires serait contre-productifs, ils finiraient par s'ennuyer de l'activité.
%Au lieu de ça, je compte les faire travailler sur un projet "fil-rouge" sur le trimestre, voire l'année. Je développerais cette idée dans le PCI.
%	* Cette solution est déjà en cours de tests, et permet de faire une synthèse plus cohérente en fin de séance, 
%ainsi qu'un lancement plus pertinent en début de séance suivante. 
%Ça m'a notamment permis de faire plus souvent les explications au groupe entier, et à mieux accorder mon temps aux élèves en difficultés.
%	* Ne pas laisser les moins rapides prendre trop de retard : faire des activités "à deux niveaux".
%Réorganiser les séances, actuellement une séance est découpée en plusieurs thématiques, proposant une série d'activité/questions.
%On pourrait proposer moins d'activité par thématique, et déplacer ces activités vers une dernière partie "aller plus loin", facultative (=réservée pour les plus rapides).
%	* Lorsque plusieurs élèves arrivent à un point délicat de l'activité, j'essaye de les regrouper autour d'un ilôt pour leur faire une explication en groupe, 
%je leur laisse un peu de temps pour assimiler, puis procède à une phase de questions/réponses, enfin, 
%je vais voir au cas par cas (ou binôme par binôme) pour déverrouiller les dernières difficultés.
%
%* À propos de la variété dans les supports
%	* Lors de l'activité visitée, les élèves disposaient d'un site internet agrégeant les ressources ainsi que 
%d'un document au format Word pour répondre à une série de questions. L'absence de variété dans le support est effectivement assez peu stimulante.
%	* Lors des activités suivantes sur la numération et le codage, les supports sont un peu plus variés : 
%recherches sur internet, utilisation de logiciels (éditeur hexa, visionneur d'images, "guide des automatismes"), calculs sur feuille.
%La difficulté de l'utilisation de maquettes à déjà été discuté plus haut, et quelques solutions ont été apportées.
%	* La diversité des supports se fera assez naturellement dans les autres disciplines.
%
\end{document}
