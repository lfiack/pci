\documentclass[pdftex,a4paper,12pt]{article}

\usepackage[francais]{babel}
\usepackage{epsfig}
\usepackage{amsfonts}
\usepackage[utf8x]{inputenc}
%\usepackage{palatino}
\usepackage{mathptmx}
\usepackage{geometry}
\usepackage{float}
\usepackage{amsmath}
\usepackage[babel=true]{csquotes}
\usepackage{graphicx}
\usepackage{algorithm}
\usepackage{algorithmic}
\usepackage{listings}
\usepackage{color}
\usepackage[squaren, Gray, cdot]{SIunits}
\usepackage[square]{natbib}
\usepackage{eurosym}
\usepackage{listings}

\geometry{a4paper,headheight=16pt,tmargin=20mm,
          bmargin=20mm,lmargin=20mm,rmargin=20mm}
%\geometry{height=23cm}
%\geometry{hmargin={3cm,2cm}}

\usepackage{caption}
\usepackage{subcaption}

\usepackage{minitoc}
%\setcounter{minitocdepth}{3}
\mtcselectlanguage{french} 

\usepackage{hyperref}

\usepackage{tabularx}
\usepackage{array}
\usepackage[normalem]{ulem}
\newcolumntype{L}[1]{>{\raggedright\let\newline\\\arraybackslash\hspace{0pt}}m{#1}}
\newcolumntype{C}[1]{>{\centering\let\newline\\\arraybackslash\hspace{0pt}}m{#1}}
\newcolumntype{R}[1]{>{\raggedleft\let\newline\\\arraybackslash\hspace{0pt}}m{#1}}

\usepackage[printonlyused,withpage]{acronym}

\newcommand{\figpath}{figures}

%Règles de dépôt : vous devrez déposer un document unique comprenant :
%une page de garde avec votre nom et votre prénom, le niveau de la classe,
%l’indication et l’intitulé de l’unité d’enseignement et l’année universitaire en
%cours. 
%Une introduction d’une quinzaine de lignes pour indiquer le contexte
%(situation géographique, effectif...) 
% La rédaction de votre analyse réflexive en
%réponse aux questions de votre tuteur ESPE (3/4 pages maximum). Des
%annexes si nécessaire.

%\sloppy
\begin{document}
	\begin{titlepage}
%~ % \newcommand{\HRule}[2]{\centering\rule{#1}{#2}}
%~ \newlength{\logowidth}
%~ \setlength{\logowidth}{2.5cm}
%~ \newlength{\logohspace}
%~ \setlength{\logohspace}{\linewidth}
%~ \addtolength{\logohspace}{-3\logowidth}
%~ 
%~ \begin{center}
%~ % Upper part of the page
%~ % \begin{minipage}{0.4\textwidth}
%~ %   \begin{flushleft} 
%~ %     \includegraphics[width=4cm]{./figure/Logo_ETIS.png}\\[1cm]
%~ %   \end{flushleft}
%~ % \end{minipage}
%~ % \begin{minipage}{0.4\textwidth}
%~ %   \begin{flushright}
%~ %    \includegraphics[width=3cm]{./figure/Logo_CNRS.png}\\[1cm]
%~ %   \end{flushright}
%~ % \end{minipage}
%~ % \begin{minipage}{0.4\textwidth}
%~ %  
%~ % \end{minipage}
    	%~ %%%%%%%%%%%% Logos
	%~ %\includegraphics[height=2.5cm]{./stage_fig/Logo_ETIS.eps}\hspace{\logohspace}
	%~ \includegraphics[height=2.5cm]{./stage_fig/Logo_ETIS}\hspace{\logohspace}
	%~ %\includegraphics[height=2.5cm]{./stage_fig/Logo_ENSEA.eps} 
	%~ \includegraphics[height=2.5cm]{./stage_fig/Logo_ENSEA} 
%~ 
%~ % \vspace{5cm}
%~ \vfill
%~ \textsc{\LARGE Rapport de stage}\\[1.5cm]
%~ % Title
%~ \HRule \\[0.4cm]
%~ { \huge \bfseries Codes LDPC non-binaires pour compression avec information adjacente}\\[0.4cm]
%~ \HRule \\[1.5cm]
%~ 
%~ % Author and supervisor
%~ \begin{minipage}{0.4\textwidth}
%~ \begin{flushleft} \large
%~ Anne \textsc{SAVARD}
%~ \end{flushleft}
%~ \end{minipage}
%~ \begin{minipage}{0.4\textwidth}
%~ \begin{flushright} \large
%~ \emph{Encadrant:} \\
%~ Claudio \textsc{WEIDMANN}
%~ \end{flushright}
%~ \end{minipage}
%~ 
%~ \vfill
%~ 
%~ % Bottom of the page
%~ {\large \today}
%~ 
	%~ %%%%%%%%%%%% Logos
	%~ \includegraphics[height=2.5cm]{./stage_fig/Logo_UCP}\hspace{\logohspace}
	%~ %\includegraphics[height=2.5cm]{./stage_fig/Logo_UCP.eps}\hspace{\logohspace}
	%~ %\includegraphics[height=2.5cm]{./stage_fig/Logo_CNRS.eps} 
	%~ \includegraphics[height=2.5cm]{./stage_fig/Logo_CNRS} 
	%~ %%%%%%%%%%%%
%~ \end{center}
\thispagestyle{empty}
\newcommand{\HRule}[2]{\centering\rule{#1}{#2}}
% \newlength{\logowidth}
% \setlength{\logowidth}{2.5cm}
% \newlength{\logohspace}
% \setlength{\logohspace}{\linewidth}
% \addtolength{\logohspace}{-3\logowidth}

Lycée Blaise Pascal
\vspace{20em}
\begin{center}
%Règles de dépôt : vous devrez déposer un document unique comprenant :
%une page de garde avec votre nom et votre prénom, le niveau de la classe,
%l’indication et l’intitulé de l’unité d’enseignement et l’année universitaire en
%cours. Une introduction d’une quinzaine de lignes pour indiquer le contexte
%(situation géographique, effectif...) La rédaction de votre analyse réflexive en
%réponse aux questions de votre tuteur ESPE (3/4 pages maximum). Des
%annexes si nécessaire.

	\HRule{\linewidth}{0.5mm}\\
	\vspace{-0.35cm}
	\HRule{\linewidth}{0.3mm}\\
	\textbf{
	\LARGE
	Analyse réflexive (2)\\
	}
	\large
	sur un séance d'Enseignements Technologiques Transversaux\\
	en première STI2D
	\HRule{\linewidth}{0.3mm}\\
	\vspace{-0.45cm}
	\HRule{\linewidth}{0.5mm}

	\vspace{10em}

	par\\
	Laurent Fiack\\

	\vspace{3em}
	pour la validation du parcours adapté en SII\\
	\vspace{3em}
	\today{}

\end{center}

\end{titlepage}    

%Une introduction d’une quinzaine de lignes pour indiquer le contexte
%(situation géographique, effectif...) 
	\section{Introduction}
	\subsection{Contexte général}
	Jeune professeur agrégé stagiaire, j'ai pris mes fonctions il y a environ deux mois au Lycée Blaise Pascal de Rouen, 
	situé au c\oe{}ur de la ville sur la rive gauche de la Seine.
	Ce lycée est assez imposant puisqu'il accueille environ 1200 élèves, près de 120 enseignants et plus de 60 personnels non-enseignants.\\
	
	Au cours de cette année de titularisation, deux classes me sont confiées.
	La première est l'une des 9 secondes où j'interviens une heure et demie en enseignement d'exploration \og{}Sciences de l'Ingénieur\fg{}.

	La seconde est l'une des 4 premières STI2D où je co-anime les séances d'applications pratiques d'Enseignements Technologiques Transversaux (abbrégés ETT) 
	avec ma collègue Nathalie Lebarbier pour un volume de 5 heures hebdomadaires.
	J'interviens également une heure par semaine en classe entière, dans la même matière, ainsi qu'une heure en accompagnement personalisé.\\

	À travers ce document, je vais tenter de développer une analyse réflexive sur la démarche pédagogique que j'ai mise en place depuis ma prise de fonction,
	en m'appuyant sur les remarques formulées par mes tuteurs terrain et ESPE au cours de la visite conseil 
	d'une des séances d'application pratique le lundi 2 octobre 2017.

	\subsection{La séance en question}
	La séance visitée était la dernière d'une séquence de trois activités d'étude de dossier pour découvrir du langage SysML.
	Au cours de cette séquence, les élèves ont d'abord découvert quelques notions par l'étude d'un scooter sous-marin lors de la première activité. 
	Ils ont ensuite approfondi ces notions dans l'étude d'une voiture télécommandée avec la deuxième activité
	pour finalement aborder tous les diagrammes du langage SysML au programme de la STI2D pendant la troisième activité.

	Cette dernière étude de dossier concerne un aménagement urbain : une borne escamotable.
	Au cours de cette activité, les élèves disposaient d'une page web agrégeant les ressources ainsi que
	d'un document au format Word pour répondre à une série de questions. 
	
	Le choix de réaliser des activités spécifiquement centrées sur le langage SysML est discutable. 
	Si ce choix a été fait, c'est par demande du collègue professeur de spécialité SIN car il se base sur ce prérequis.
	À mon sens, il serait plus judicieux de distiller la découverte du SysML par l'étude de systèmes tout au long de l'année 
	dans le cadre d'activités sur d'autre thématiques.

	\subsection{Les points à travailler}
	Parmis les remarques formulées, quatre ont été mise en avant par mes tuteurs:
	\begin{enumerate}
		\item Au cours de la séance visitée, les élèves étaient chacun face à un ordinateur et travaillent avec un questionnaire, des ressource et le document réponse. 
			Pourquoi ne pas les faire travailler en binôme ?
		\item Organisation de la fiche professeur de préparation de séance, listant de manière plus où moins détaillée l'ensemble des points à aborder.
		\item Ajouter plus de concret, plus d'interactivité classe/professeur, et varier les supports.
		\item Approfondir et tester une idée d'activité \og{}fil rouge\fg{} sur l'année ou le trimestre pour les élèves finissant le travail en avance.
	\end{enumerate}

	Dans cette analyse, je me concentrerais sur les premier et troisème points, qui s'articulent autour de la gestion de la classe 
	et dont les éléments de réponse me semblent complémentaires.

	Le point numéro 2 quant à lui concerne l'organisation des séances et évolue au fil des séances.
	Enfin, le quatrième point mérite à mon sens une étude sur un plus long terme, je souhaite donc le développer au cours de mon \og{}Projet Collaboratif Innovant\fg{}.

	\section{Travailler en monôme ou en binôme ?}
	\subsection{Une question d'organisation}
%* Dans la salle, il y a un ordinateur par élève, les élèves se mettent naturellement chacun devant une machine. Il se sont placés par affinité.
	La salle d'activité pratique est organisée en ilôts de quatre postes informatiques, et d'autant de chaises.
	Il y a suffisamment de postes pour que chaque élève puisse avoir accès au sien.
	Les élèves se sont naturellement placés devant une machine, l'organisation dans les ilôts s'est faite par affinité.
%* Il y a 4 machines par ilôt, donc max 4 élèves. Les élèves au sein d'un ilôt ont chacun leur machine, mais peuvent facilement communiquer (et y sont autorisés).
	Bien que chaque élève soit seul en face de sa machine, ils peuvent aisément communiquer entre eux au sein d'un ilôt, et y sont autorisés.\\

%* Les premières séances ont consisté en des études de dossier ayant pour objectif de découvrir le langage SysML.
%Ces travaux ne nécessitent pas de matériel, hormis les systèmes étudiés dans les différentes activités.
%Ce matériel n'est pas présent en nombre suffisant pour que chaque ilôt ou binôme dispose du sien.
	Les premières séances ont consisté en des études de dossier ayant pour objectif de découvrir le langage SysML.
	Ces travaux ne nécessitent pas de matériel, hormis les systèmes étudiés dans les différentes activités.
	Ce matériel n'est pas disponible en nombre suffisant pour que chaque ilôt ou binôme dispose du sien et circule donc dans la classe pendant l'activité.

%	* Les activités suivantes, ayant eu lieu entre la visite et la rédaction de ce document, abordent les notion de codage de l'information et du numération.
%Ces activités ne nécessitent pas plus de matériel et les élèves travaillent toujours face à leur machine.
%	* Lors des séances futures, en énergie notamment, les élèves seront amenés à plutôt manipuler les maquettes en groupe.
	Les activités suivantes, ayant eu lieu entre la visite et la rédaction de ce document, abordent les notion de codage de l'information et du numération.
	Ces activités ne nécessitent pas plus de matériel et les élèves travaillent toujours face à leur machine.
	Lors des séances futures, en énergie notamment, les élèves seront amenés à plutôt manipuler les maquettes en groupe.\\

%* Lors d'une séance, nous sommes deux professeurs pour une classe entière. 
%Je m'adapte aux méthodes de ma collègue, qui agit selont le fonctionnement de l'établissement.
	Au cours de la séance, je m'adapte aux méthodes de ma collègue Nathalise Lebarbier, avec qui je co-anime la séance.
	En tant que nouvel arrivant, je m'adapte à sa manière de travailler, fidèle au fonctionnement de l'établissement.

	\subsection{Le travail en monôme et l'autonomie}

%Elle préfère voir les élèves travailler en monome, pour plusieurs raisons :
%	* La quantité de bruit est moins élevée. L'augmentation naturelle du volume sonore des élèves qui travaillent en groupe incite les élèves à parler encore plus fort. 
%Lorsqu'ils travaillent en binôme, il faut régulièrement les rapeller à l'ordre.
%	* L'évaluation, par compétences ou non, est plus aisée.
%	* Lors du baccalauréat, les élèves seront seuls face à leur copie. 
%Même si cette affirmation n'est pas fausse en soit, je ne suis pas tout à fait d'accord avec cet argument: 
%même s'il reste un objectif indispensable, le bac n'est, à mon sens, pas le seul objectif de la formation: 
%La formation vise également à engager les élèves dans une poursuite d'études, pour qu'ils puissent, par la suite, s'intégrer dans le monde de l'entreprise.
%Les jeunes bacheliers seront alors bien évidemment amenés à collaborer avec leurs pairs, d'abord dans l'enseignement supérieur, 
%et à plus forte raison dans leur carrière professionnelle.
	Au cours de discution avec mes collègues, plusieurs motifs sont énoncés pour justifier le travail en autonomie des élèves.
	D'abord, de manière assez pragmatique, la quantité de bruit dans la salle est moins élevée. 
	L'augmentation naturelle du volume sonore des élèves qui travaillent en groupe incite leurs camarades à parler encore plus fort, phénomène qui s'amplifie rapidement. 
	Lorsqu'ils travaillent en binômes, il faut régulièrement rapeller les élèves à l'ordre.

	Ensuite, quand les élèves travaillent seuls, il est plus aisé d'évaluer leur niveau et de remédier à des difficultés. 
	Au contraire, lorsqu'ils travaillent en groupe, certaines difficultés peuvent passer inaperçu.

	Enfin, lorsqu'ils passeront les épreuves du baccalauréat, les élèves seront seuls face à leur copie.
	Même si cette affirmation n'est pas fausse en soit, j'aimerais ajouter un eu de nuances à cet argument.
	Même s'il reste un objectif indispensable, le bac n'est, à mon sens, pas le seul objectif de la formation.
	La formation vise en effet également à engager les élèves dans une poursuite d'études, pour qu'ils puissent, par la suite, s'insérer dans le monde de l'entreprise.
	Les jeunes bacheliers seront alors bien évidemment amenés à collaborer avec leurs pairs, 
	d'abord dans l'enseignement supérieur, et à plus forte raison dans leur carrière professionnelle.\\

%* La mise en binôme de deux élèves est souvent bénéfique. Les deux élèves n'ont pas les mêmes niveaux de compréhension dans tous les dommaines.
%L'élève ayant un meilleur niveau de compréhension des notions abordées par l'activité en cours va souvent ré-expliquer ces notions à son binôme, 
%les rôles pouvant s'inverser en fonction des activités.
%Cette reformulation est bénéfique pour les deux membres, d'une part un élève ayant des difficultés pourra tirer bénéfice de la reformulation d'un de ces camarades.
%D'autre part, l'élève qui reformule les notions s'assure de mieux les comprendre et de mieux les retenir.
%* Il faut toutefois veiller, notamment lors d'une grande différence de niveaux, que l'élève ayant le plus de facilité ne réalise pas le travail à la place de son camarade.
%* En faisant travailler les élèves en monôme, j'ai pu constater que certains élèves ne sont pas familiarisé à l'utilisation de systèmes informatiques. 
%Je préfère m'assurer qu'ils aient tous l'opportunité de le manipuler, car je suis convaincu que ce n'est qu'en manipulant qu'ils pourront mieux s'y familiariser.

	\subsection{Le travail en groupe et la collaboration}
%Même s'il a été admis que, dans le monde de l'entreprise, on est amené à collaborer avec ses pairs, il est rare, dans l'industrie de travailler en binôme sur un même poste. Les élèves doivent ainsi, selon moi, apprendre à collaborer, tout en fournissant un travail personnel.


\end{document}
