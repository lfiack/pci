\documentclass[pdftex,a4paper,12pt]{report}

\usepackage[francais]{babel}
\usepackage{epsfig}
\usepackage{amsfonts}
\usepackage[utf8x]{inputenc}
%\usepackage{palatino}
\usepackage{mathptmx}
\usepackage{geometry}
\usepackage{float}
\usepackage{amsmath}
\usepackage[babel=true]{csquotes}
\usepackage{graphicx}
\usepackage{algorithm}
\usepackage{algorithmic}
\usepackage{listings}
\usepackage{color}
\usepackage[squaren, Gray, cdot]{SIunits}
\usepackage[square]{natbib}
\usepackage{eurosym}
\usepackage{listings}

\geometry{a4paper,headheight=16pt,tmargin=20mm,
          bmargin=20mm,lmargin=20mm,rmargin=20mm}
%\geometry{height=23cm}
%\geometry{hmargin={3cm,2cm}}

\usepackage{caption}
\usepackage{subcaption}

\usepackage{minitoc}
\setcounter{minitocdepth}{3}
\mtcselectlanguage{french} 

\usepackage{hyperref}

\usepackage{tabularx}
\usepackage{array}
\usepackage[normalem]{ulem}
\newcolumntype{L}[1]{>{\raggedright\let\newline\\\arraybackslash\hspace{0pt}}m{#1}}
\newcolumntype{C}[1]{>{\centering\let\newline\\\arraybackslash\hspace{0pt}}m{#1}}
\newcolumntype{R}[1]{>{\raggedleft\let\newline\\\arraybackslash\hspace{0pt}}m{#1}}

\usepackage[printonlyused,withpage]{acronym}

\newcommand{\figpath}{figures}

%Règles de dépôt : vous devrez déposer un document unique comprenant :
%une page de garde avec votre nom et votre prénom, le niveau de la classe,
%l’indication et l’intitulé de l’unité d’enseignement et l’année universitaire en
%cours. 
%Une introduction d’une quinzaine de lignes pour indiquer le contexte
%(situation géographique, effectif...) 
% La rédaction de votre analyse réflexive en
%réponse aux questions de votre tuteur ESPE (3/4 pages maximum). Des
%annexes si nécessaire.

%\sloppy
\begin{document}
	\begin{titlepage}
%~ % \newcommand{\HRule}[2]{\centering\rule{#1}{#2}}
%~ \newlength{\logowidth}
%~ \setlength{\logowidth}{2.5cm}
%~ \newlength{\logohspace}
%~ \setlength{\logohspace}{\linewidth}
%~ \addtolength{\logohspace}{-3\logowidth}
%~ 
%~ \begin{center}
%~ % Upper part of the page
%~ % \begin{minipage}{0.4\textwidth}
%~ %   \begin{flushleft} 
%~ %     \includegraphics[width=4cm]{./figure/Logo_ETIS.png}\\[1cm]
%~ %   \end{flushleft}
%~ % \end{minipage}
%~ % \begin{minipage}{0.4\textwidth}
%~ %   \begin{flushright}
%~ %    \includegraphics[width=3cm]{./figure/Logo_CNRS.png}\\[1cm]
%~ %   \end{flushright}
%~ % \end{minipage}
%~ % \begin{minipage}{0.4\textwidth}
%~ %  
%~ % \end{minipage}
    	%~ %%%%%%%%%%%% Logos
	%~ %\includegraphics[height=2.5cm]{./stage_fig/Logo_ETIS.eps}\hspace{\logohspace}
	%~ \includegraphics[height=2.5cm]{./stage_fig/Logo_ETIS}\hspace{\logohspace}
	%~ %\includegraphics[height=2.5cm]{./stage_fig/Logo_ENSEA.eps} 
	%~ \includegraphics[height=2.5cm]{./stage_fig/Logo_ENSEA} 
%~ 
%~ % \vspace{5cm}
%~ \vfill
%~ \textsc{\LARGE Rapport de stage}\\[1.5cm]
%~ % Title
%~ \HRule \\[0.4cm]
%~ { \huge \bfseries Codes LDPC non-binaires pour compression avec information adjacente}\\[0.4cm]
%~ \HRule \\[1.5cm]
%~ 
%~ % Author and supervisor
%~ \begin{minipage}{0.4\textwidth}
%~ \begin{flushleft} \large
%~ Anne \textsc{SAVARD}
%~ \end{flushleft}
%~ \end{minipage}
%~ \begin{minipage}{0.4\textwidth}
%~ \begin{flushright} \large
%~ \emph{Encadrant:} \\
%~ Claudio \textsc{WEIDMANN}
%~ \end{flushright}
%~ \end{minipage}
%~ 
%~ \vfill
%~ 
%~ % Bottom of the page
%~ {\large \today}
%~ 
	%~ %%%%%%%%%%%% Logos
	%~ \includegraphics[height=2.5cm]{./stage_fig/Logo_UCP}\hspace{\logohspace}
	%~ %\includegraphics[height=2.5cm]{./stage_fig/Logo_UCP.eps}\hspace{\logohspace}
	%~ %\includegraphics[height=2.5cm]{./stage_fig/Logo_CNRS.eps} 
	%~ \includegraphics[height=2.5cm]{./stage_fig/Logo_CNRS} 
	%~ %%%%%%%%%%%%
%~ \end{center}
\thispagestyle{empty}
\newcommand{\HRule}[2]{\centering\rule{#1}{#2}}
% \newlength{\logowidth}
% \setlength{\logowidth}{2.5cm}
% \newlength{\logohspace}
% \setlength{\logohspace}{\linewidth}
% \addtolength{\logohspace}{-3\logowidth}

Lycée Blaise Pascal
\vspace{20em}
\begin{center}
%Règles de dépôt : vous devrez déposer un document unique comprenant :
%une page de garde avec votre nom et votre prénom, le niveau de la classe,
%l’indication et l’intitulé de l’unité d’enseignement et l’année universitaire en
%cours. Une introduction d’une quinzaine de lignes pour indiquer le contexte
%(situation géographique, effectif...) La rédaction de votre analyse réflexive en
%réponse aux questions de votre tuteur ESPE (3/4 pages maximum). Des
%annexes si nécessaire.

	\HRule{\linewidth}{0.5mm}\\
	\vspace{-0.35cm}
	\HRule{\linewidth}{0.3mm}\\
	\textbf{
	\LARGE
	Analyse réflexive (2)\\
	}
	\large
	sur un séance d'Enseignements Technologiques Transversaux\\
	en première STI2D
	\HRule{\linewidth}{0.3mm}\\
	\vspace{-0.45cm}
	\HRule{\linewidth}{0.5mm}

	\vspace{10em}

	par\\
	Laurent Fiack\\

	\vspace{3em}
	pour la validation du parcours adapté en SII\\
	\vspace{3em}
	\today{}

\end{center}

\end{titlepage}    

%Une introduction d’une quinzaine de lignes pour indiquer le contexte
%(situation géographique, effectif...) 
	Jeune professeur agrégé stagiaire, j'ai pris mes fonctions il y a environ deux mois au Lycée Blaise Pascal de Rouen, 
	situé au c\oe{}ur de la ville sur la rive gauche de la Seine.
	Ce lycée est assez imposant puisqu'il accueille environ 1200 élèves, près de 120 enseignants et plus de 60 personnels non-enseignants.\\
	
	Au cours de cette année de titularisation, deux classes me sont confiées.
	La première est l'une des 9 secondes où j'interviens une heure et demie en enseignement d'exploration \og{}Sciences de l'Ingénieur\fg{}.

	La seconde est l'une des 4 premières STI2D où je co-anime les séances d'applications pratiques d'Enseignements Technologiques Transversaux (abbrégés ETT) 
	avec ma collègue Nathalie Lebarbier pour un volume de 5 heures hebdomadaires.
	J'interviens également une heure par semaine en classe entière, dans la même matière, ainsi qu'une heure en accompagnement personalisé.\\

	À travers ce document, je vais tenter de développer une analyse réflexive sur la démarche pédagogique que j'ai mise en place depuis ma prise de fonction,
	en m'appuyant sur les remarques formulées par mes tuteurs terrain et ESPE au cours de la visite conseil du lundi 2 octobre 2017.\\
\end{document}
